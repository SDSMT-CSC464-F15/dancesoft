% !TEX root = DesignDocument.tex


\chapter{Project Overview}
This section provides some information with regards to the 
team roles, project management, and phase overview.[MB]

\section{Team Members and Roles}
The DanceSoft team consist of two members, Marcus Berger and Dicheng Wu.

Marcus Berger(Scrum Master/Development Team) -  As a member of a two person team the roles for this project blend together significantly. Both team members mostly do equal shares of all work types. As the primary manager of the DanceSoft Trello board, Marcus had mostly taken on the role of scrum master within the group. However his primary role is still development of DanceSoft.

Dicheng Wu(Development Team) - Dicheng's primary role as with both members of the team, is development of the software. However like the other members since the team is so small each member of the two person team must be able to fill all needed roles within the team.

Dr. Jeff Mcgough(Product Owner) - While not a working member of the team Dr. Mcgough is a secondary scrum master and product owner to the group due the small size of the team. Main duties in this role include talking to the client about what is needed, and making sure the team stays on task and is going in the correct direction based on the clients needs .

Author notation will be identified by [MB] for Marcus and [DW] for Dicheng at the start of
major sections (ex: after heading 2.0 above). Subsections within the main sections assumed to be by the
same author. [MB] 


\section{Project Management Approach}
The client requires a weekly meeting every Wednesday at 2:00 p.m. to check on the progress of the system during the fall 2015 semester at South Dakota School of Mines and Technology. During the spring 2016 semester the team meeting changed to Tuesdays at 2:00 p.m. and Thursdays at 3:00 p.m. if necessary. These meeting vary on topic and length depending on the needs of the project and the status of task. The senior design class requires that this project be completed in six sprint that are all roughly three weeks long, with a week long results period after each one. Another project requirement is the presentations that are required by the senior design class. These presentation occurs twice every semester usually after the first and last sprint each semester. Each presentation cover the content of the project up to that point, and updates on topics such as risk mitigation, budget, and current prototypes. Lastly it was requested by Dr. McGough as part of senior design and as the client that we provide him with access to the Github repository for the project and the Trello board for check in purposes.

The internal team management was mostly was mostly done by Marcus Berger and Dr. McGough. Marcus used the free service Trello to manage the tasks that the team needed to complete. As the project progressed the team had to reevaluate the management approach as the DanceSofts project requirements changed. After sprint 3 it was decided that due to time left for the project that the management of the student interface would need to be finish in a future iteration.    


\section{Stakeholder Information}

The stakeholder and sponsoring customer for this project is Dr. Jeff McGough a computer science professor at the South Dakota School of Mines and Technology and the vice president of the Academy of Dance Arts in Rapid City, South Dakota.


Well not the sponsor of the project Dr. McGough's wife, Julie McFarland is also a key part of the customer base and a stakeholder as the owner and artistic director of the Academy. Other academy members while not directly related also share a stock in the project development as the software directly processes and modify data given to the academy. Students and teachers will not have the access to direct evaluation of the various iterations of the DanceSoft project. However as the project nears completion in future senior design classes or however the clients choose to proceed, these group could be effected by the uses of and updates to the software. Therefore they have a secondary stake in the projects various iterations. 


\subsection{Customer or End User (Product Owner)}
The primary end user for this product is Julie McFarland and her employees to manage the Academy of Dance Arts. Julie well not playing a direct role in product development is able to convey the academy's needs through Dr. McGough. Dr. Jeff McGough is the primary point of contact in the project. He assumes the role of scrum master at times and drives the product backlog while providing more details on product backlog materials during the weekly meeting with the development team. 

\subsection{Management or Instructor (Scrum Master)}
Dr. Jeff McGough is the primary point of contact in the project. He assumes the role of scrum master at times and drives the product backlog and provides more details on product backlog materials during the weekly meeting with the development team. He also acts as the assessment for project progress. This means that as the project progressed Dr. McGough is able to reevaluate and reestablish requirements as the project needed or was requested by the client. 


\subsection{Developers --Testers}
The DanceSoft team consist of two members, Marcus Berger and Dicheng Wu, who are both primarily developers and testers. Due to the fact that the team is only two members, all development roles are shared between the two team members. The teams job is to develop, test, fix and adapt the various request and requirements given by the client/scrum master Dr. McGough. As developers the team takes the user stories and develops a backlog for each sprint. Then the teams job is to take these user stories and produce the code needed to complete each backlog entry.

After the backlog entry has been complete it is the developers job to test and error check the functions to make sure that the function completes all its needed tasks, in the way the team wants. Also testing is done to make sure the files don't create any problems with the rest of the software. Testing is covered in more detail in the testing section of this document.

\section{Budget}
There was no budget or monetary compensation for this project. When the project was initially laid out there was an idea for a budget that would be used for a Linux box. The Linux box would have been used to store the system on a local device. However as the requirement were brought down and redefined to an iteration of the project, rather then a full complete project, it became clear that the project would not reach a deployment state where a local machine was necessary. However the client provided a linux box for the senior design fair which was not claimed as a budget by this project. As work continues on this project the budget may grow.  
Monetary compensation for this project follow the guidelines laid out in the DanceSoft Software Contract. The team will be given and accept no form of compensation for the work on the DanceSoft or any related project. This is in accordance with the conditions of the project, client, and the senior design class.

\section{Intellectual Property and Licensing}
The intellectual property for this project is currently the property of the client Jeff McGough. However due to Dr. McGough's involvement with the South Dakota of Mines and Technology, and the senior design class, the South Dakota Board of Regents policies must be considered. These policies require that any work done by a teacher or for a teacher by a group of one or more students be submitted as property of the South Dakota Board of Regents.

The only way to avoid this policy and prevent the Academy software from being owned by the state is to make the project open source in accordance with the board's policy. As a result of this set of policies all code worked on by the DanceSoft team is stored in a public Github repository provided by SDSM&T for the senior design class.

Another intellectual property and licensing policy that must be considered as part of the teams choice to use PyQt is the GPL (general public license), under which the PyQt software falls. This license requires that the user release all source code for a project using any GPL licensed code in their software. However this is only required if the client plans on distributing the software they have created. As such this license will not cause any issues for Dr. McGough or the Academy of Dance Arts since the client does not plan on distributing the code to anyone other then themselves. 

Based on the goals of the project the team has constructed as software that does not violate any intellectual property rules or regulations. Therefore the current rights to the software belong solely to the client until such a time as the client choices to distribute the project for monetary gain.  

\section{Sprint  Overview}
This project will be implemented in phases, the phases follow. [MB]

\subsection{Sprint 1: Initial User Story and Requirements Gathering}
This phase consisted of meetings and discussing the user stories and requirements with the product owner. The product owner also laid out limitations and constraints for the project. More information can be found in section \bf 3.0 Requirements \rm

Also tackled during this phase was the research conducted into the tools the team were going to use for the project. There were three main areas the team needed to tackle to decide the frameworks for the project. First the team needed to pick the programming language that would be used for the project. Two main languages were analyzed, Python and Swift. Swift is the Mac native language recently released by Apple Inc. the benefits found for Swift were its nativity to the Mac OSX and it would be the easiest to integrate into Mac. However since Swift was relatively new the team would have to take even more time to learn the new language, and since the client did not know Swift it would be harder to analyze and make adaptions to in the future. Python had the advantage of being a language that is more universally known, both to the team, the client, and the online community that the team could turn to for support or questions. Due to this and a few other factors laid out in more detail in \bf Sprint Report 1 \rm the team choose Python as the programming language for the projects primary programming language.

The other two areas of research were database and interface framework. The team looked to PyQt, Kivi, Tkinter, and other graphical interface frameworks. The team decided after looking at each one that due to the ease of use and some companies listing GUI experience, that the team would use PyQt as the framework for the interface. Then the DanceSoft team looked into database software. After looking at both SQL and non-SQL datebase options, such as MySQL and Mongo the team went with a MySQL database for the storage. As with programming language, the research results are laid out in \bf Sprint Report 1 \rm   

\subsection{Sprint 2: Database Creation and Starting Pages}
During this phase the database schema was constructed and implemented, being sure to keep the schema as concise as possible. The goal was to generate a database that could effectively support the needed functionality and GUI connectivity. After the database was constructed, the team moved on to the starting landing pages which are jumping off pages for functionality creation. These page are modified as phases progress and further functionality was added.

During this phase the team also plotted out the first drafts of the overall user interface flow for the desktop GUI. The team later modified this as the projects requirements were flushed out. The team would also reexamine this structure in later sprints as certain requirements were dropped, added or modified. Once the first verison of the database was up and running within the School of Mines' MySQL sever the team began working on the log in page, and the main navigation pages for both Academy admins and teachers.

The log in page is used to confirm the user of the system and check whether or not the user has admin permissions. This system is required because of the different features the admins can access within the system, such as changing pay rates or adding students or teachers to the system. The log in page then directs to the landing page selection where the user, if they have permission, can access the admin or teacher interfaces.

The first drafts of the admin and teacher interfaces were implemented in this phase as well. The admin interface contained buttons through which the user can access the various functionality of the interface, These include "Manage Employees" which takes the user to most of the employee related features within the project, also the "Manage Classes", "Manage Students", and "Manage Billing and Payroll" buttons take the user to the class, student, and billing and payroll features respectfully. The teacher interface like the admin interface contains buttons for managing students and classes, however these features differ slightly between interfaces since admins have more system access. The teacher interface also contains a "Personal Information" button which allows the user to modify certain aspects of their information in the system, such as their hours, or their user-name and password. 

\subsection{Sprint 3: Functionality Creation }
This phase will compose the bulk of the project as the first development of the functionality requested by the project owner are constructed. As the prototype progresses check ins with the client will be conducted to be as sure as possible that the team stays on track. Pages will be tested as the pages are constructed.

\subsection{Sprint 4: Development Updates and Payroll/Billing}
This phase will consist of updates to existing functions, updates, and prototyping updated, new, or misunderstood functions. Testing will be conducted as updates are made to ensure new pages work, and updated pages retain requested function. During this phase user stories and requirements should move to final completion.

\subsection{Sprint 5: Updates Bug Fixes and Functionality}
During this phase the last bit of testing on the remaining functionality will be conducted. After this the final production needs will be completed, this may consist of final deployment or other steps depending on the state of the project at this time. Lastly the project will be handed over to the client for delivery and the senior design fair. By this point the product itself will live on the box provided by the client and a production server. Access Information, logs, and user guides and manuals will be provided, with the user guide as both pdf files and psychical copies.

\subsection{Sprint 6: Updates, Fixes, and Iteration Packaging}


\section{Terminology and Acronyms}
\begin{enumerate}
\item Backlog - A list of task to be completed
\item Budget - money provided by the client to supply the needed materials for the project
\item Database - A storage and platform to manipulate data for the projects 
\item GUI - Graphical User Interface - the front end screen that the users interact with using graphical assets
\item Sprint - three week time periods where portions of the project are completed
\item Timeline - the plan of when a assignment is to be completed
\item GPL - General Public License - a type of software development license
\item SDSM&T - the South Dakota School of Mines and Technology's university acronym
\item SDBOR - South Dakota Board of Regents
\item Interface - The screen or set of screens that the user can see and navigate within the project
\item Source Control - A system used to manage multiple people working on the same collections of infromation to maintain consistency,  
\end{enumerate}

\section{Sprint Schedule}
The sprint schedule.  Can be tables or graphs.   This can be a list of dates with the visual 
representation given below.

\section{Timeline}
Gantt chart or other type of visual representation of the project timeline.

\section{Backlogs}
Place the sprint backlogs here.    The product backlog will be in the chapter with the user 
stories.   


\section{Development Environment}
The basic purpose for this section is to give a developer all of the necessary information to setup their development environment to run, test, and/or continue development of the DanceSoft project, while also providing information on the systems the team used to create the project.

\subsection{Source Code Development}
The source code for this project was written in Python. Due to the cross platform nature of this language the team used a variety of different IDEs over the course of development, these included:

\begin{enumerate}
\item Python IDLE
\item Microsoft Visual Studio 2015
\item Xcode 
\end{enumerate}

Any developers wishing to continue work on this project should be able to run this software within any python capable interface of their choice.  

\subsection{MySQL Database}
The projects back end contains a MySQL database. MySQL can be installed on a developers computer free of charge off of the MySQL website. This package also contains a helpful GUI interface called MySQL workbench which can aid the user in manipulating the database if they so choose. However the system should be editable within any MySQL enabled database manipulation software. Once a developer has MySQL installed they simply need to log in and connect to the database using credentials provided by the client.

\subsection{PyQt}

\section{Development IDE and Tools}
Describe which IDE and provide links to installs and/or reference material. 

\section{Source  Control}
Source control for this project was conducted using Github, and the Github GUI. Github is a git add in, a developer  
could use whatever git manager they want. The Github GUI can be installed from www.github.com. The repository was provided by South Dakota School of Mines as part of the senior design class. A developer wishing to continue work on this project simply need a git software and access as a contributor to the repository. However once the clients move the code to a local device a developer will need access to this device as well.
 
\section{Dependencies}
Describe all dependencies associated with developing the system. 

\section{Build  Environment}
A user, or developer can build the project through the use of precompiled python scripts. A user can run the log in script which will compile some of the Python scripts as to speed up various aspects of running the project. There are no special build scripts that are requires before the project can be run.

\section{Development Machine Setup}
If warranted, provide a list of steps and details associated with setting up a 
machine for use by a developer. 


