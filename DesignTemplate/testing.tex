% !TEX root = DesignDocument.tex


\chapter{System  and Unit Testing}

This section describes the approach taken with regard to testing the DanceSoft project. 

\section{Overview}
Provides a brief overview of the testing approach, testing frameworks, and general 
how testing is/will be done to provide a measure of success for the system.

Each requirement (user story component) should be tested.    A review of objectives and
constraints might be needed here.  

\section{Dependencies}
Describe the basic dependencies which should include unit testing frameworks and 
reference material. 


\section{Test Setup and Execution}
Describe how test cases were developed, setup, and executed.  This section can 
be extremely involved if a complete list of test cases was warranted for the system.   One 
approach is to list each requirement, module, or component and describe the test.

The unit tests are described here.

\section{System Testing}

\section{System Integration Analysis}

\section{Risk Analysis}
During development of the DanceSoft project the project encountered a few possible risk that could effect the system. The team has found the following main risk within the system.

\begin{enumerate}
\item Interface Usability
\item Database Connectivity
\item Data Security
\item Data Backup
\end{enumerate}

The first of the main risk is the PyQt interface usability. The system must maintain a simple and easy to use interface. This way the users can efficient and effectively access the system while maintaining the functionality at the systems core. The team must always keep this fact in mind when developing any of the systems pages and structures, as the users at the Academy are not assumed to have a technical background.

The second risk is the database connectivity. If the database can not be reached the project is incapable of accomplishing its functions. The system at its core is a database manipulation software, without the ability to connect to the database the project can not function. The system must therefore assure connection before allowing the users to continue.

The third set of risk with the DanceSoft system is database backups and data security. Database backups can be achieved by using functions within the MySQL database software. These backups can then be stored on a local system or wherever the client chooses to store the data. Database security can be achieved using prepared statements and queries, and other techniques of encryption and protection.  


\subsection{Risk Mitigation}
The following section is the teams approaches, goals, and ideas to mitigate the various risk with this the DanceSoft project.

The teams has been developing each interface with the idea of usability in mind during the entire process. One of the testing approaches that the team used was user testing. This method meant sitting down and acting like a user of the system and not a developer or technical user. If during this process something did not make sense or execution did not work in the expected way, the interface or function was reworked. Another part of this approach involved having people do sudo-beta test by sitting and attempting navigation of the system. If any questions or concerns came up from the user they were analyzed and addressed if necessary by the team. By keeping the interface simple and asking for feedback the team hopes to mitigate usability issues as much as possible.

The database risk are only partially dealt with in the current project. The database connection issues are handled through a few error checks in the connection. Also since the database will be contained in the local machine provided by the client the database can be connected to directly. This connection issue will become greater once a student web interface is added to system, or if the client moves the database the connection will become reliance on the communication between either the two systems or the network connectivity at the Academy. Some of these future student interface issues were discussed before the student portal was dropped from the project. The rink could possibly be handled by storing the information the students are sending through a socket and storing it until the database is booted at the Academy. At which point the updates would be submitted.
Another database risk is the security of the data stored within. The backup issue can be solved through writing the contents to a file which can be stored on either the local machine, or another machine as to protect the data should something happen to the computer in which the database is stored. Many MySQL management software also have ways of exporting the statements needed to create the database which the client can use to manage backups and create text file backups. The actually security of the data will not be completed in this iteration of the software. However the team has a few ideas for data security. Firstly since only the local system exist at this time, data should not be passed over a network. The SQL queries can be placed into prepared statement to aid in their security. Its is highly advised by the development team that security solutions be explored by future developers as the current team lacks the experience and expertise to truly ensure the security of the data contain within the system.

\section{Successes, Issues and Problems}

\subsection{Changes to the Backlog}
Overall the project's base remained unchanged from the the beginning requirements. However as the project was being built the backlog and project encountered a few changes and iterations. These changes and modification are briefly explained in this section. 

During the project the backlog was changed multiple times. The first of these redesigns occurred after sprint 2. The team and the client elaborated on some of the user stories to give the team an increased amount of focus for the various functions. This allowed the team to better develop the core system functionality required by the project.

However the first major change to the project's backlog occurred after sprint 3.5. It became clear to the team that the student web portal would not be completed by the end of the senior design project period. After discussions with the client in order to make the local desktop GUI as complete as possible the student interface was dropped from the project requirements. Users stories for this aspect of the project included: student registration, view tuition, and student schedules. In response to these changes the team ported the functions into the local desktop interface to create the student registration, billing, schedule, and other interfaces.

The last changes to the backlog occurred in the last sprint of the project. The project had been refined at this point as a first iteration of the DanceSoft project. This change became necessary as issues arose within the team that prevented the team from reaching full project completion. As a consequence of this the project became a proof of concept or minimum viable product for the client. The team demoed the existing software for the client, at this meeting the client and the team discussed the requirements again and reevaluated the project requirements for completion, and the senior design fair presentation. 

