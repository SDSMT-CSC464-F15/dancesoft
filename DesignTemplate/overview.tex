% !TEX root = DesignDocument.tex


\chapter{Overview and concept of operations}

This chapter contains a  general overview of the DanceSoft project. [MB]

\section{Team Members and Team Name}

The DanceSoft Team consist of:
\begin{enumerate}
\item Marcus Berger
\item Dicheng Wu
\end{enumerate}


\section{Client}
The stakeholder and sponsoring customer for this project is Dr. Jeff McGough a computer science professor at the South Dakota School of Mines and Technology and the vice president of the Academy of Dance Arts in Rapid City, South Dakota.
Well not the sponsor of the project Dr. McGough's wife, Julie McFarland is also a key part of the customer base and a stakeholder as the owner and artistic director of the Academy of Dance Arts in Rapid City, South Dakota. The clients main goal is to use the software to manage the day to day activities of the Academy, and as the number of iterations of the project increase add students and refines the project. 



\section{Project}
This first iteration of the 

\subsection{Purpose of the System}
The purpose of this system is to provide a system for the Academy of Dance Arts to manage their day to day operations, their employees, and their students. These day to day operations range from assigning teachers to classes, looking up information, printing roles sheets for classes, etc.  Also the system must accomplish these task using a user friendly interface, that is as intuitive as possible.


\section{Business Need}
The customer needs us to develop a software solution which can run the dance studio in an effective manner. The product also needs to handle changing classes from year to year without needing to be updated. This means that the software needs to sync with multiple users, and handle new information such as class rosters, prices, clothing requirements for classes, changes in the employment roster, and many other changes that can occur in the running of a dance school.\\
This project as a whole needs to be an improvement on the current system in use by the customer and provide an easy and efficient way to run the clients business. This project will  the data manipulation task through the back-end MySQL database, and the ease of use will be handled with a simple Pyqt GUI [MB]

\bf(NEED TO ADD MORE CLARITY WILL ADD TO AND ADAPT AFTER SEEING CURRENT INTERFACE)
   

\section{Deliverables} 
Listed below are the deliverable major system componets for this project. [MB]

\subsection{Major System Component: Database}
The first major component of the system is the MySQL relational database. The database contains the Academy's data and is the core of the back-end side of the software. This database will be the conduit for most of the systems interactions with the data. The database will live within a local computer provided by the user.  

\subsection{Major System Component: User Interface}
The second major component of the system is the front-end user interface for admins and teachers. This will be the only part of the system most users ever see, and will provide an effective means to complete the desired user task. This is accomplished through the use of pages created in PyQt with interfaces to give users effective ways to interact with the database and the necessary data for their requested operations.

\section{Systems Goals}
The system needs to provide a solution which can run the dance studio data and some day to day activities in an effective and secure manner. This includes allowing teachers to print role sheets, look at schedules, and manage their information. Students need to have the ability to see information pertinent to them such as registration and class requirements. Owners need to be able to use the system to manage their employees, the academy's students and it classes, and other administrative duties such as billing and payroll  Lastly this system as a whole also needs to be an improvement on the current system in use by the customer and provide an easier and more efficient way to run the clients business.

Overall the system goal is to provide a environment where academy owners, teachers, and students can effectively manage their personal needs and requirements for academy participation and continued operations.

\section{System Overview and Diagram}
\textmd{Users will access this application through a local GUI or a web application depending on their position within the system. The GUI will be located in the academy local box, and the student web application will be excess able through any web browser.
When a user makes a connection to the website, the pages and data will be sent to the user. Figure 1 shows an overview of the student interface architecture.
When a local GUI user connects the user will navigate through the vareous pages listed above to the desired functionality. Figure 2 shows a simplistic view of the GUI Architecture.
There are three major components to the project, database, student interface, and admin/teacher interface.  Each section is described in more functional detail above and in} \bf section 4 Design and Implementation.
\bf (ADD MORE DETAIL AS SYSTEM FINALIZES) [MB]


\bf(ADD FIGURES AFTER SYSTEM DESIGN FINALIZATION)


\section{Technologies Overview}
\textmd{The primary Technologies for this projects are as follows:}

\begin{enumerate}
\item Xcode and Visual Studio - Xcode and MSVS are the primary development IDEs for this project. Of the two the one the group used the most was visual studio so we could develop on the machines provided by the school.
\item Python - the primary programming language for the project
\item PHP - primary language for student web interface
\item PyQt and Qt Designer - Gui package and development environment 
\item MySQL - MySQL provides the database and relational quires to manage the data and organize it within the system.
\end{enumerate}


\textmd{These technologies were selected after a first research sprint where research into programming language, GUI, and database options were selected. A brief description of the research can be found in the sprint 1 report or the first prototype sections of this document. [MB]}

