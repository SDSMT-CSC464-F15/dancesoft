% !TEX root = SystemTemplate.tex
\chapter{User Stories, Backlog and Requirements}
\section{Overview}


The overview should take the form of an executive summary.  Give the reader a feel 
for the purpose of the document, what is contained in the document, and an idea 
of the purpose for the system or product. 

 The userstories 
are provided by the stakeholders.  You will create he backlogs and the requirements, and document here.  
This chapter should contain 
details about each of the requirements and how the requirements are or will be 
satisfied in the design and implementation of the system.

Below:   list, describe, and define the requirements in this chapter.  
There could be any number of sub-sections to help provide the necessary level of 
detail. 





\subsection{Scope}

This section covers the purpose of systems, the client's information and restrictions on the system, and the user stories for requirements.

What scope does this document cover?  This document would contain stakeholder information, 
initial user stories, requirements, proof of concept results, and various research 
task results. 



\subsection{Purpose of the System}
The purpose of this system is to provide a system for the Academy of Dance Arts to manage their day to day operations, their employees, and their s friendly students. These day to day operations range from assigning teachers to classes, looking up information, printing roles sheets for classes, etc.  Also the system must accomplish these task using a user friendly interface, that is as intuitive as possible. Students will also have a simple web interface to look up their information and register for new classes.


\section{ Stakeholder Information}

The stakeholder and sponsoring customer for this project is Dr. Jeff McGough a computer science professor at the South Dakota School of Mines and Technology and the vice president of the Academy of Dance Arts in Rapid City, South Dakota.
Well not the sponsor of the project Dr. McGough's wife, Julie McFarland is also a key part of the customer base and a stakeholder as the owner of the Academy. 


\subsection{Customer or End User (Product Owner)}
The primary end user for this product is Julie McFarland and her employees to manage the Academy of Dance Arts. Julie well not playing a direct role in product development is able to convey the academy's needs through Dr. McGough.

Dr. Jeff McGough is the primary point of contact in the project. He assumes the role of scrum master at times and drives the product backlog and provides more details on product backlog materials during the weekly meeting with the development team. 

\subsection{Management or Instructor (Scrum Master)}
Dr. Jeff McGough is the primary point of contact in the project. He assumes the role of scrum master at times and drives the product backlog and provides more details on product backlog materials during the weekly meeting with the development team.


\subsection{Developers --Testers}
The DanceSoft team consist of two members, Marcus Berger and Dicheng Wu, who are both primarily developers and testers. due to the fact that the team is so small. All development roles are shared between the two team members.


\section{Business Need}
Use this section to define what business need exist and how this software will 
meet and/or exceed that business need.   

\section{Requirements and Design Constraints}
Use this section to discuss what requirements exist that deal with meeting the 
business need.  These requirements might equate to design constraints which can 
take the form of system, network, and/or user constraints.  Examples:  Windows 
Server only, iOS only, slow network constraints, or no offline, local storage capabilities. 


\subsection{System  Requirements}
What are they?  How will they impact the potential design?  Are there alternatives? 


\subsection{Network Requirements}
What are they? 


\subsection{Development Environment Requirements}
What are they?  Is the system supposed to be cross-platform? 


\subsection{Project  Management Methodology}
The stakeholders might restrict how the project implementation will be managed. 
 There may be constraints on when design meetings will take place.  There might 
be restrictions on how often progress reports need to be provided and to whom. 
 
\begin{itemize}
\item What system will be used to keep track of the backlogs and sprint status?
\item Will all parties have access to the Sprint and Product Backlogs?
\item How many Sprints will encompass this particular project?
\item How long are the Sprint Cycles?
\item Are there restrictions on source control? 
\end{itemize}

\section{User Stories}
After the requirement for the project were laid out the team created the user stories based on those requirement. The user stories the team came up with are as fallows:

\begin{enumerate}
  \item As a user i want to adjust students payment models
  \item As the owner I would like to see automatic database backups.
  \item As a student I would like to be able to register online (with special app). Classes must be approved before added.
  \item As a student I would like to be able to search clothing requirements.
  \item As a student I would like to know my billing.
  \item As the owner I would like to indicate clothing requirements per class.
  \item As a studio person, I would like to be able to add students to classes.
  \item  As a student, teacher etc, I would like to be able to look up a students class list.
  \item As the teacher I would like to get a class role for each class.
  \item Given a class list, I would like to get an invoice of the tuition due.
  \item Studio would like to track payments and estimate remainder due.  I would like to generate an invoice for this amount.
  \item As a student I would like to be able to register online (with special app).   Classes must be approved before added.
  \item As a student I would like to know my billing.
  \item As the owner I would like to track teacher hours and compute payroll.
  \item As the owner I would like to indicate clothing requirements per class.
  \item As a student I would like to be able to search clothing requirements.
  \item As the owner I would like to see automatic database backups.
\end{enumerate}



\subsection{User Story \#1}
User story \#1 discussed. 

\subsubsection{User Story \#1 Breakdown}
Does the first user story need some division into smaller, consumable parts by 
the reader?  This does not need to go to the level of actual task definition and 
may not be required. 

\subsection{User Story \#2} 

\subsubsection{User Story \#2 Breakdown}
User story \#2  .... 

\subsection{User Story \#3} 

\subsubsection{User Story \#3 Breakdown}
User story \#3  .... 


\section{Research or Proof of Concept Results}
This section is reserved for the discussion centered on any research that needed 
to take place before full system design.  The research efforts may have led to 
the need to actually provide a proof of concept for approval by the stakeholders. 
 The proof of concept might even go to the extent of a user interface design or 
mockups. 


\section{Supporting Material}


This document might contain references or supporting material which should be documented 
and discussed  either here if appropriate or more often in the appendices at the end.  This material may have been provided by the stakeholders  
or it may be material garnered from research tasks.

